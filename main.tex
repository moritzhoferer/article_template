\documentclass[
	12pt, % allowed font sizes : 10pt (default), 11pt and 12pt
	a4paper, % paper sizes: a4paper (default), a5paper
	%draft, %draft mode: faster compilation. figures not loaded and shows overfull boxes
	%twocolumn, % multicolumn: onecolumn (default), two column text
	%fleqn, % left-aligned formulas
	% leqno, % formula labels on the left-hand side
	%landscape % landscape mode
	]{article}

% JUST ADD PACKAGES YOU REALLY NEED
\usepackage{amssymb} % advanced math symbols
\usepackage{amsmath} % for equations,  math functions and operators
\usepackage{amsfonts} % extended fonts
\usepackage{authblk}
%\usepackage{eurosym} % Euro symbol
\usepackage{geometry} % adjust margins below with \geometry{options}
%\usepackage{graphicx} % \includegraphics preferable as PDF or PNG
%\usepackage{subfigure} % subfigures in subfigure environment
%\usepackage{caption} % captions for figures and tables
%\usepackage{color} % color text
\usepackage{setspace} % adjust line spacing with \onehalfspacing, \doublespacing and \singlespacing
%\usepackage{footmisc} % adjust footnotes
\usepackage{natbib} % advanced bibliography management
%\usepackage{pdflscape} % used for singular landscape pages
\usepackage{array} % visualizes matrices 
\usepackage[breaklinks=true, colorlinks=true, linkcolor=black, citecolor=black,]{hyperref} % set hyperlinks in the document and to the outside (web-pages, mail-addresses)
\usepackage[amsmath,hyperref]{ntheorem} % replacement for  amsthm to use \theoremstyle{break}
\usepackage{lineno} % show line numbers in the left margin
\linenumbers

\onehalfspacing
\theoremstyle{break} % options: break (default), definition, plain, remark
% Here you can definition your own theorem environments. DON'T USE A HIERARCHY!
\newtheorem{definition}{Definition}
\newtheorem{assumption}{Assumption}
\newtheorem{theorem}{Theorem}
\newtheorem{corollary}{Corollary}
\newtheorem{lemma}{Lemma}
\newtheorem{proposition}{Proposition}

% Always make a linespace in front of every proof.
\newenvironment{proof}[1][Proof]{\noindent\textbf{#1.}}{\ \rule{0.5em}{0.5em}}

% Use \C{width}, \L{width} and \R{width} in tabular environment to make tables with fixed column width. C, L, and R stand for center, align left and align right
\newcolumntype{L}[1]{>{\raggedright\let\newline\\arraybackslash\hspace{0pt}}m{#1}}
\newcolumntype{C}[1]{>{\centering\let\newline\\arraybackslash\hspace{0pt}}m{#1}}
\newcolumntype{R}[1]{>{\raggedleft\let\newline\\arraybackslash\hspace{0pt}}m{#1}}

% adjust your margins here
\geometry{left=2.54cm, right=2.54cm, top=2.54cm, bottom=2.54cm}

\begin{document}

\begin{titlepage}
\singlespacing
\title{
	The Title of your Work\thanks{Be nice!}
}

\newcommand{\mailto}[1]{\thanks{\href{mailto:#1}{#1}}}
\renewcommand\Affilfont{\small}


% Add the authors.
\author[a]{Hans Gersbach\mailto{hgersbach@ethz.ch}}
\author[a]{Name Surname\mailto{yourmail@ethz.ch}}
\affil[a]{Center~of~Economic~Research, ETH~Zurich, Z\"{u}richbergstrasse~18, 8092~Zurich, Switzerland}

\date{Last updated: \today}

\maketitle

\begin{abstract}
	\noindent
	How to write an abstract\footnote{
		Copied from \cite{boyce2010global} with comments and a note form \textit{Nature Education}:
		With just under 200 words, this abstract can convey the motivation for and outcome of the work with some accuracy, without intimidating readers by its length.
		If the journal allows to have the abstract as multiple paragraphs, start a new one before the findings.
	}: \textbf{(Context)} In the oceans, ubiquitous microscopic phototrophs (phytolankton) account for approximately half the production of organic matter on Earth, thus affecting the abundance and diversity of marine organisms and strongly influencing
	climate processes.
	\textbf{(What we have)} Analyses of the satellite-derived phytoplankton concentration (available since 1979) have suggested decadal fluctuations linked to climate forcing, but the length of this record is insufficient to resolve longer-term trends.
	\textbf{(What we want)} To estimate the time dependence of phytoplankton biomass since the beginning of oceanographic measurements in 1899, 
	\textbf{(Task)}	we combined available ocean transparency measurements and in situ chlorophyll observations.
	\textbf{(Object of the document)} This paper presents the trends we identified at local, regional, and global scales.
	\textbf{(Findings)} We observed declines in eight out of ten ocean regions, and estimated a global rate of decline of ∼1\% of the global median
	per year. Our analyses further revealed interannual to decadal phytoplankton fluctuations superimposed on long-term trends.
	These fluctuations are strongly correlated with basin-scale climate indices, whereas the long-term declining trends are	related to increasing sea surface temperatures.
	\textbf{(Conclusion)} In conclusion, global phytoplankton concentration has definitely declined over the past century; 
	\textbf{(Perspectives)} this decline will need to be considered	in future studies of marine ecosystems, geochemical cycling,
	ocean circulation, and fisheries.\\
	% !!! no additional linespaces here !!!
	\vspace{0in}\\
	\noindent\textbf{Keywords:} key1, key2, key3\\
	\vspace{0in}\\
	\noindent\textbf{JEL Classification:} code1, code2, code3\\
	\bigskip
\end{abstract}
\thispagestyle{empty}
\end{titlepage}

\pagebreak \newpage

\setcounter{page}{2}
\onehalfspacing

\section{Example text}\label{sec:examples}
\subsection{Sentence ends with formula followed by new paragraph}
Donec pede justo, fringilla vel, aliquet nec, vulputate eget, arcu. In enim justo, rhoncus ut, imperdiet a, venenatis vitae, justo. 

Nullam dictum felis eu pede mollis pretium. Integer tincidunt. Cras dapibus. Vivamus elementum semper nisi. Aenean vulputate eleifend tellus
\begin{equation}
\exp\{-i\pi\} + 1 = 0.
\end{equation}

Aenean leo ligula, porttitor eu, consequat vitae, eleifend ac, enim. Aliquam lorem ante, dapibus in, viverra quis, feugiat a, tellus.
\begin{theorem}[Optional theorem title]
	Nam eget dui. Etiam rhoncus. Maecenas tempus, tellus eget condimentum rhoncus, sem quam semper libero, sit amet adipiscing sem neque sed ipsum.
\end{theorem} Nam quam nunc, blandit vel, luctus pulvinar, hendrerit id, lorem.

\subsection{Ongoing sentence  with sequence of formulas}
Lorem ipsum dolor sit amet, consectetuer adipiscing elit. Aenean commodo ligula eget dolor. Aenean massa. Cum sociis natoque penatibus et magnis dis parturient montes, nascetur ridiculus mus. Donec quam felis, ultricies nec
\begin{equation}
\exp\{-i\pi\} + 1 = 0,
\end{equation}
pellentesque eu
\begin{equation}
E = m c^2,
\end{equation}
pretium quis, sem. Nulla consequat massa quis enim. Donec pede justo, fringilla vel, aliquet nec, vulputate eget, arcu. In enim justo, rhoncus ut, imperdiet a, venenatis vitae, justo.

\section{Introduction} \label{sec:introduction}
Motivate your work! Define the gap your contribution closes.

\section{Literature Review} \label{sec:literature}
Always cite your favorite authors, like \cite{gersbach2017regulation}.

\section{Model} \label{sec:model}
\begin{assumption}
	Whatever you want!
\end{assumption}

\section{Analysis} \label{sec:analysis}
Some introductory text...
\begin{proposition}\label{prop:label}
	content...
\end{proposition}

\section{Conclusion} \label{sec:conclusion}
Wrap it up.
Summarize the issue.
Highlight your contribution again.
Introduce possible extensions.

\singlespacing

% That's the necessary style for economics publications
\bibliographystyle{apalike}
% You can add several .bib-file with comma as separator
\bibliography{references}

\clearpage

\doublespacing
\appendix
\section{Proofs} \label{app:proofs}
The Proofs start on a new page.
This section gives the extensive proofs to preceding propositions.

\begin{proof}[Proof to Proposition \ref{prop:label}]
	Go hard!
\end{proof}
\end{document}
